\begin{flushleft}
    {\fontsize{16}{18}\selectfont\textbf{Essay Four: The influence of language on human cognition}} 

    \bigskip

    {\fontsize{14}{16}\selectfont \textit{Lecture 9: “Researching humans though their languages”}}
\end{flushleft}

\section*{Summary}
When we study languages we might ask ourselves if they are able to influence human cognition. Such question was answered by researchers who highlighted the actual effects of the language we speak on many cognitive aspects, overall how language shape the mental representation that each individual has of the external world, as the Sapir–Whorf hypothesis suggests \parencite{Koerner_1992}. One of the main evidence given to this hypothesis uses colors datasets, demonstrating that cross-linguistic differences influence on color discrimination from memory \parencite{Cibelli_2016}.

The researcher Damián Blasi has been studying the relationship between linguistic diversity and human cognition, considering cultural, evolutionary, anthropology and computational science data, to determine the consequences that the structure of languages have on human cognition. One of his main questions is: “is it true that the better the structure of a language is the better is the representation of it in the world?” \\ 
As a premise to such question and related crosse-linguistic topics, he points out the disproportionate reliance that cognitive science have on English, losing important evidence on all the other languages that can help us better understand these phenomena \parencite{Blasi_2022}. Additionally, in a study treating neurodegeneration, he and colleagues showed how the lacking cross-linguistic research is fundamental to detect neural marker of neurodegenerative diseases \parencite{Garcia_2023}. 

Furthermore, Blasi talked about his other lines of research, which include the study of languages history and transformation, as well as their surviving. Languages go through a natural transformation, due to to geopolitical events and consequent influence with other surrounding languages, which as highlighted in this talk is inevitable: every existing language has been affected by others, even the ones of aboriginal communities. This includes also surviving languages, which have an apparently simpler grammatical structure due to their contamination with simpler languages themselves.

Finally, Blasi mentioned the future outcomes and risks of the human-machine era, where we use constantly language with non-human entities; and the probability of a decrease of the number spoken languages, which would culminate in a uniformity of languages, as well as in a decline of human cognition diversity.

\section*{Personal view}
Participating at this talk was not only intriguing and stimulating, but also decisive for me. During the years of my Bachelor I found a profound interest for social and cultural aspects of psychology: I've always wanted to grasp which are the cultural factors that lead us to certain acts and beliefs. Thus, I chose to write my bachelor's thesis on emotional lexicon in a cross-linguistic perspective; my goal was to understand if speaking a particular language would have influenced the emotions we feel or their intensity. \\
In some languages there are many different terms that represent a particular feeling, whilst in others that same feeling has none: I wondered if people who don't have a word to describe an emotion don't feel it either, and if individuals who speak a language with various terms to represent an emotion are also able to experience it in those different ways.

During the writing of my thesis I tried to read up as much as I could, what I immediately found out was that this topic is very broad and complex, more than I thought it would: a lot of diverse disciplines tried to give an answer to these questions, with very distinct points of view, generating theories and objections. \\
This topic is addressed by psychologists, linguists and philosophers, everyone with their own arguments, occasionally conflicting. The main contrast about this matter stands between theories supporting the universality of emotions, i.e. the existence of the same innate emotions for everyone, independently of the language or culture \parencite[for example]{Izard_2010}; and the socio-constructivism approach, which states that emotions are a product of the social environment we live in, our culture and, consequently, the language we speak \parencite[for example]{Barbara_1996}. 

Towards the end of my thesis I was able to grasp some of the inter-linguistic difference present in emotional lexicon, which are mainly differentiated by emotional dimensions, that can be found in the semantic organization of emotional lexicon. The ones present in most models being: cognitive and physiological arousal and hedonic dimension (i.e. valence) \parencite{Russell_1980}. \\
These results were really interesting but, at the end I couldn't really comprehend the conclusions of such differences, I wasn't able to see the actual differences that language had on emotions nor on the cognition. \\
The talk that Blasi gave was able to expand my past research, giving more evidence  and explanations, especially for what concerns the cognitive aspects for cross-linguistic differences. \\
I could reconnect other studies I read about language shaping mental representation of the world: for example, Inuit have many distinct words to define snow, due to the environment they are surrounded by. 

Later, Blasi talked  about language transformation, history and diversity, which were pretty new to me. My attention was always more focused on human cognition, and I never actually thought of these matter, I knew how much language influence one another and the transformation that they have during time, but I never thought seriously on how much this could influence human cognition too. \\
With the globalization, Blasi pointed out the risks of language diversity and the influence that this could have on cognition diversity: personally, I see this happening, even while writing this essay and taking a master in a different language than mine, but I wouldn't be too keen on saying that this will have a strong impact on diversity cognition. Of course I don't have any evidence to state this, but I think that the environment around us will always constraint us to differentiate the language we use in order to better represent it. 

To conclude, I would say that we should protect and cherish our diverse culture and language, also by studying other languages and people far from us: learning about why we use disparate terms and the influence that these have on our behavior and thoughts is always fascinating and can really enrich our experience as humans. 
\begin{flushleft}
    {\fontsize{16}{18}\selectfont\textbf{Essay One: gender inequality in science}} 

    \bigskip

    {\fontsize{14}{16}\selectfont \textit{Lecture 3: “Picture a Scientist”}}
\end{flushleft}

\section*{Summary}
The 2020 documentary “Picture a Scientist” \parencite{picture} shows in a real and moving way the inequality of women in science, underlying the many difficulties that bright researchers such as the biologist Nancy Hopkins, the chemist Raychelle Burks and the geomorphologist Jane Willenbring had to face during their careers because of their gender. 
Several interviews with these and other researchers have taken place to describe from a first-person point of view the huge gap that has been present for decades for women in STEM: the protagonists go through their worst experiences of gender inequality such as sexual harassment, being treated as inferior, being paid less and being given less equipment and space for their research. 

Burks focuses on racial discrimination too, despite having to face different treatment for being a woman in science, she also had to overcome many difficulties for being black: she shows some statistical reports, from which it can be seen that the percentage of white men in STEM is 47.9\%, of white women is 25.7\%, whilst for black women just 2.2\%. 
This describes a unique challenge for women of color trying to make their way into the scientific research field; Bruks tells about her experience in university where none of her professors were black women, or about how many times she was not taken into consideration when speaking in conferences or being confused with a janitor, even if sitting at her desk, wearing her coat. \\
This can make us understand the number of obstacles that a woman can encounter when trying to find a job, to make a career, and how many resources are required to get what she wants and deserves. 

Women in science have come together to solve this huge issue, they tried to get better conditions, equal treatment and
One of the pioneers in the feminist battle in science is the biologist Nancy Hopkins, who worked as a professor and researcher at MIT during the 70s; there she started experiencing gender discrimination in different ways. In the interview she speaks about the allocation of her workspace: her laboratory was smaller than the ones of her male colleagues, but her request for a bigger space was denied.  
To prove the actual size difference of the lab, Hopkins decided to measure the spaces during the night to make her point with concrete data: this was the beginning of her battle for equal rights for female researchers. Hopkins was able to get together the few women researchers at MIT to speak up about the abuses that they were all suffering, being paid less, not being taken as seriously as their male collogues and being sexually harassed by men in the same environment; organizing a committee to fix, or at least try to fix the problem. 

During the decades the number of women in science has grown, with the acquisition of more equal gender conditions in terms of salary, opportunities and treatment. This is encouraging, but when we look at the statistics we see that the percentages are growing slowly (from 7\% in 1970 to 29\% in 2017) and that sexual harassment and unequal job opportunities are still taking place, which is unacceptable. 

\section*{Personal view}
I think this is a very well-made documentary, that everybody working in this field and not should watch to grasp the challenges that women had to face in the past, which could lead us to get better conditions nowadays, and some that women are still facing.\\
What we can learn are also the numbers that demonstrate the actual gender gap present in STEM, with a higher inequality when the level of the career rises: in fact, if we look at Bachelor's degrees we can find 50\% of women, but as we take into consideration higher positions in academia such as postdocs or employed figures, we can find a percentage of women between 30 and 35. This is the so-called "glass ceiling", which describes the decreasing number of women in higher positions in society, representing huge difficulties and discrimination for women to get prestigious and relevant jobs in a patriarchal society.
I think that this is a very relevant issue, that leads to missing opportunities for society itself to have great minds, with bright and innovative ideas, that can make important changes in every field. 

If we want to investigate the causes of such gender impairment, I think we would have to go from the ancestral times, when women needed to procreate, educate children and cook to more recent ones, where they are asked to do the exact same thing. I think that on many occasions, closed-minded people, who don't have much space for change, still see women as less capable than men of making a career in many fields, one of these being science. This way of perceiving women leads to abuses of every kind, disrespect and big gender gaps. 

Describing the problem like this is not sufficient, we have to take into account also the important cognitive aspects that stand behind it. Several social experiments have been conducted to understand if we have brain mechanisms that reproduce gender stereotypes and if this can be diminished in some ways. What has been found is that we all have strong biased associations for what concerns gender inequality, which are hard to erase. 
Many studies have been made with I.A.T. (Implicit Associations Test), i.e. studying the unconscious associations that stand between two words: if these have a strong connection, we will be faster to put them together. Usually in the experiments, some words are shown and the participants have to associate them with another word, that represents a category. 
Studying associations, allows us to understand which stereotypes we have and in which measure they are rooted in our brain. \\
In the documentary, Doctor Mahzarin Banaji, a social psychologist, explains the usage of this test with scientific careers and family related nouns that had to be associated with female and male names. When she administered this test what she found was a huge delay in the response when people needed to associate science career names with female ones and family names with male ones. What surprised her the most was that she, a woman in science, had a big difficulty in doing this test as well. \\
This demonstrates how much gender stereotypes are rooted in our cognitive system and, consequently, how much effort is required to overcome them. 

To conclude, I think that we need to really engage ourselves to diminish gender stereotypes: we have to start with little girls, encouraging them to study scientific subjects and telling them they have the same potential and giving them the same opportunities as boys to become great scientists; we need to have more representation in the media, showing women in power, with the white coat and googles making discoveries that can change the world. \\
We need to take all the knowledge and the possibilities we've got to have as many women in science as men, so that the next time we are asked to picture a scientist what will pop up into our mind will be, finally, a woman, of any ethnicity.
\begin{flushleft}
    {\fontsize{16}{18}\selectfont\textbf{Essay Two: processing differences between consonants and vowels}} 

    \bigskip

    {\fontsize{14}{16}\selectfont \textit{Lecture 4: “Functional differences between Cs and Vs”}}
\end{flushleft}

\section*{Summary}
When we perceive language our brain is processing many different linguistic features, that will allow us to understand correctly the meaning of a sentence. A relevant phenomenon that has been studied in this field concerns the functional differences that consonants and vowels hold during language processing. 

One of the principle discoveries that has been made is that we tend to give more weight to consonants rather than vowels to access the lexicon: this has been proven with anecdotal evidence, as well as with neurological evidence. 
We can see this phenomenon in our everyday life, for example in brand names that remove vowels and leave just the constants, or in tongue twisters: we can notice that the lexical information relies in the consonants; in fact, even if we eliminate the vowels the information is still being conveyed. \\
Another fact that allow us to understand the importance of consonants compared to vowels is the different distribution of them in languages: it has been discovered that there isn't any language that has more consonants than vowels; with just two exceptions of Danish, where there seems to be an equal distribution and the peculiar case of Hawaiian. 
Such evidence is also supported by the fact that there are languages that have lexical roots composed only by consonants, but none composed just by vowels. 

If we look at neurological evidence, we can observe that vowels and consonants are processed as distinct elements by our brain: they activate different regions, they are represented independently and categorically \parencite{Caramazza_2000}. Such strong distinction plays a crucial role in determining the prosodic structure of speech, in the organization of syllables and allow lexicon-semantic processing \parencite{Carreiras_2008, Boatman_1997}. \\
Neuroimaging studies proved also the different weight that we assign to Cs and Vs: for example, incorrect responses more often involved a vowel change than a consonant change. These findings show that vowel identity is underdefined compared to consonants \parencite{Ooijen_1996}.

The contexts in which functional differences of consonants and vowels processing have been observed are many: in priming effects of transported letters, that are produced only when they are consonants \parencite{Parea_2004}; in word learning, where it has been determined a first preference for Vs rather than Cs in newborns, a tendency that seems to reverse from the first year of life \parencite{Nazzi_2015}. \\
Such vowel bias (opposite from what we have discussed until now) has been observed in animals too: experimental evidence reveals that rats, just like newborns are better at discriminating vowels than consonants \parencite{Toro_2019}. Further studies have tried to observe if domestic dogs, that are more exposed to human language than rats, would present a consonant bias as humans or not. On the contrary of what expected, what has been observed is that also dogs possess a vowel bias rather than a consonant bias, resulting that the consonant bias may be a human-exclusive tool for language processing \parencite{Mallikarjun_2021}. 

We should also mention the distinct contributions of Cs and Vs in language learning: what has been found is that consonants seem to contribute to lexical processing of words, whilst vowels are primarily concerned with the extraction of structural generalization \parencite{Bonatti_2005, Toro_2008}. 

\section*{Personal view}
Learning about this topic sparked a strong curiosity in me: I found it really interesting, especially examining it under the light of  many perspectives: the human brain (during the different development stages), the animal brain and finally the machine system. \\ 
I come from a strictly psychological background so for me almost everything about the linguistic world is new, especially if we study language processing in a detailed manner; thus listening to the lecture and afterwards reading through the articles was really fascinating.

The thought that kept sticking with me while studying the topic and writing this essay is of how many brain processes we are completely unaware of, that are able to make us interact with the world around us. I will take myself as an example: I'm italian and I am pretty fluent in English and Spanish, but even knowing the vowels that I use in these languages, I never thought of how quantitatively less they were compared to the consonants, or the roles that they hold that are able to make me process each one of these languages; nor the complex processes that my brain had to make to extrapolate the lexical, syntactical or grammatical features in order to make me comprehend each word of a sentence and allow me to communicate with other people. \\
I know that this may seem obvious, but I think that studying cognitive science and linguistic, as well as many other subjects in a research perspective, is the key to understand how we and the world around us works: asking ourselves questions about every detail is the engine for curiosity and knowledge. 

What made me appreciate the topic was also the possibility to take a look into the animals brain: we usually don't think that animals can process language, nor that they may have higher cognitive abilities as we do. Therefore it surprised me that rats and dogs showed a vowel bias in language processing, just like infants do. This proved that the consonants bias is strictly human, but also that animals have a preference for one of these key language elements as well, just like infants, who have been exposed to language for a limited time. 

Eventually, I would have to say that the field in which such Cs and Vs differences have been studied, that has got me the most curious is the one of artificial intelligence: it has been demonstrated that ChatGPT, the large language model of OpenAI, displays consonant bias, just as humans do. It tends to identify words using consonants rather than vowels \parencite{Toro_2023}. 
This is just one of the many human biases that this chatbot present (i.e. some papers demonstrated the presence of gender bias): this reveals how the human training of the model influences its outcomes and how powerful human biases are. 

To conclude, I think that functional differences between Cs and Vs are really interesting: they make us understand how our brain is able to process and comprehend language using these elements, which biases we display and have in common with animals and machines. \\
This topic leaves me also with some questions regarding the origin of such differences as well as the ones that determined the different quantity of Vs and Cs in almost every language and the strong preference for one rather than the other. 
\begin{flushleft}
    {\fontsize{16}{18}\selectfont\textbf{Essay Three: animals and humans cognition}} 

    \bigskip

    {\fontsize{14}{16}\selectfont \textit{Lecture 7: “Animals minds”}}
\end{flushleft}

\section*{Summary}
An issue that has always been relevant for humans is to grasp the essential elements that distinguish them from animals, finding out what makes us special. In recent times we investigated possible similarities and differences between animals and humans minds through cognitive science. 

The principle element that we think of when we try to differentiate humans from animals is language; different research lines have been carried out on this topic.\\
In humans it's possible to see a strong brain specialization in the left hemisphere of the brain for what concerns language, from the moment of birth \parencite{Pena_2003}; what is interesting is that such lateralization was found in dogs brains too, especially for biological relevant sounds \parencite{Andics_2016}. \\ 
This finding is also related with the cognitive ability of using abstract rules: humans are able to remember them, which allows them to use a language correctly; it has been demonstrated that also animals (e.g. ducks) possess abstract concepts such as shape, color, but also same and different. In this studies the ability to learn of animals has been highlighted, too \parencite{Martinho_2016}. 

For what interests the field of music, was demonstrated the presence of universal principles, independent of the cultural environment of the individuals \parencite{Mehr_2019}. 
These involve features that relax or activate a person and the preference for simple frequency ratio rather than infrequent rhythmic ratio. This preference was observed also in birds \parencite{Doolittle_2014} and in lemurs, who have a bias towards small integer ratios, defining another cognitive similarity with humans \parencite{Degregorio_2021}.

Another topic that was covered by animals cognition regards their numerical abilities: an experiment with monkeys verified their skill of spontaneous quantity discrimination for grater quantity up until 3 units \parencite{Hauser_2000}; other evidence showed that fish  are able to discriminate quantity with large ratio differences \parencite{Agrillo_2008}, whilst bees can distinguish quantity if it doesn't surpass 4 units or the number isn't greater than 6 \parencite{Gross_2009}. \\
Animals own many arithmetic abilities, too: for example, different experiments made with bees \parencite{Giurfa_2022} and newborn chicks \parencite{Rugani_2015} showed that they present a mental number line (i.e. the representation of numbers from the left with the smallest towards the right), which is a universal concept in humans. \\
Furthermore, bees demonstrated to understand the concept of zero \parencite{Howard_2018}, meaning learning that 0 represents nothing, the absence of something.

Diving into other cognitive skills of animals we can observe a great episodic memory: a study made with scrub jays revealed that they have the capacity to recover episodic memories from the past \parencite{Clayton_1998}, likewise plan the future \parencite{Caroline_2007}, proving their ability of mental time travel. \\
Finally we should mention the competence of tools usage by many different animals. Research highlighted the skills of using existing tools by monkeys, but a great difficulty for them to create new ones; contrarily a good use and manufacturing of new tools was observed in New Caledonian crows and Goffin's cockatoos \parencite{Auersperg_2016,Rutz_2016}.  

\section*{Personal view}
Personally, I found this subject of study mind-opening: the question of what we have so special to consider ourselves "better" than animals crossed my mind several times, but I never actually gave an answer to it. \\
Looking at all the topics we went through to study animals minds, I would say that the one that arose the most interest in me was animals social cognition, I've always been fascinated by social cognition, and I think it might be a research line I'd persuade in a future PhD. 

Humans are often defined as "social animals" and I was curious to know which cognitive processes animals that live in social contexts have in common with humans. \\
We could learn that Pinyon jays are able to make transitive inference (i.e. they can infer new relationships from the ones they know): this seems to be possible thanks to their highly-social living environment \parencite{Bond_2003}. \\
Research has also showed the presence of mirror neurons in many animals and consequently their ability to learn from others: they can understand the intentions of others and communicate with each other: e.g. bees perform a waggle dance to confer the location to a food source \parencite{Dong_2023}. \\
What one can argue is that humans are also able to attribute mental states to others, but it was demonstrated that chimpanzees and scrub jays can do it too: they can grasp the behavior and mental conditions of others and adapt consequently \parencite{Call_2008}.

Social cognitive abilities are even more in animals that are domesticated: dogs are able to ask humans for social cues and then use them to accomplish different task that they struggled with \parencite{Here_2002}. Furthermore, during time, they went through biological changes to better intermediate with humans: they changed their facial muscles to produce expression that can trigger humans \parencite{Kaminski_2019} and they became able to produce positive loop of oxytocin when they exchange gaze with their owners \parencite{Nagasawa_2015}.

With these discoveries we are able to see how social groups deeply influence both humans and animals behaviors, as well as their cognitive skills. \\
A question that was left unanswered about this topic concerns emotion and theory of mind: are animals able to infer emotional state to their fellows, other than intentions? From what we could learn, seeing their ability to form internal society and to communicate between each other my answer would be yes, but this would be an hard topic to study. \\
In fact, the only thing that had me puzzled while reading the cited papers was the methodology used: it must be really arduous to study animals cognition, especially if we want to examine them in a non-experimental context but rather a natural one (which I think it would be best). I also believe that external variables are harder to be excluded, especially when we have to deal with training procedures: in that case we might not observe the real skills of animals, but just some potential ones inferred with human intervention.  

Despite these ambiguities, I consider the obtained results really intriguing, and I believe that these discoveries should make us ponder deeply on how we treat every kind of animals, too: maybe learning how cognitively similar they are with humans may raise awareness on their wellbeing and our responsibility to protect the environment.